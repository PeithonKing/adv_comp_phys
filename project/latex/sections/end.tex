\section{Future Work}

While this study has made significant progress in developing a Graph Neural Network-based approach for point cloud classification, there are several avenues for future exploration.

\begin{enumerate}
\item Completing the instance segmentation task by extending the current framework to identify individual objects within the point cloud is a natural next step.
\item Running the same code on a larger dataset will help to generalize the results and increase the model's robustness.
\item Calculating proper metrics such as accuracy after prediction will enable a more comprehensive evaluation of the model's performance.
\end{enumerate}

\section{Conclusion}
In conclusion, this project has been a valuable learning experience, providing insights into various aspects of point cloud processing and Graph Neural Networks (GNNs). Through this project, I had the opportunity to explore and learn about several new concepts and techniques.

\begin{enumerate}
\item Optimized searching algorithms like Octree and KDTree, which enabled efficient querying and processing of large point cloud datasets.
\item This project marked my first foray into GNNs, and I gained hands-on experience in designing and implementing a GNN-based approach for point cloud classification.
\item Working with huge point cloud data presented several challenges, and I learned effective strategies for handling and processing such datasets.
\item Additionally, I explored the concept of Siamese networks and their applications in computer vision tasks.
\end{enumerate}

I would like to express my sincere gratitude to Prof. Subhankar Mishra and Prof. Subhashish Basak, for giving me the opportunity to work on this project, Mr. Jyotirmaya Shivottam for his immence guidance and Mr. Rhishav Das and Ms. Ipsita Rout for their support throughout this project. Their valuable insights and feedback were instrumental in shaping the project's outcome.