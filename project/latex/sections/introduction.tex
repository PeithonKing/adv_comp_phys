\section{Introduction}

    Substructure-finding in dark matter halos from gravity-only N-body simulations is an important problem in astrophysics and cosmology. These simulations generate large datasets that represent the distribution of dark matter particles in the universe, which are used to study the formation and evolution of galaxies and other structures. The substructure-finding problem involves identifying and classifying halos within these simulations based on their density and spatial distribution.
    
    Traditionally, percolation algorithms have been used to solve this problem by connecting neighbouring particles in a simulation to form clusters or halos. However, these methods require simulation history and increase the simulation cost due to the need for multiple time steps. We apply point-cloud segmentation techniques using Graph Neural Network (GNN) based methods to address these limitations.
